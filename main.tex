\documentclass{llncs}

\usepackage{macros}
\usepackage{lstcoq}
\usepackage{lstocaml}
\usepackage{mathpartir}
\usepackage{amsfonts}
\usepackage{amsmath}
\usepackage{stmaryrd}
\usepackage{graphics}
\usepackage{array}
\bibliographystyle{plain}
\pagestyle{plain}

\title{Mechanizing small-inversions}
% % \subtitlte{A new elimination tactic to rule them all}
\author{Pierre Boutillier \and Thomas Braibant}
\institute{Inria Paris Rocquencourt}
\begin{document}
\maketitle
\begin{abstract}
  Inductively defined relations are pervasives in the large scale
  formalizations; and many proof steps require to invert an
  hypothesis. Coq's built-in tactic ``inversion'' is made for this
  purpose, but has severe usability and efficiency issues. 
  
  We propose a inversion tactic that mechanizes the ideas behind
  the ``small inversions'' proof-trick \cite{monin} and argue that it
  builds the right elimination principle.  
\end{abstract}
\section*{Introduction}

An inductive definition is often presetend as a set of rules (called
constructors). For example, let us define \coqe{even} by the rules
\newcommand\even{\mathtt{even}}
\begin{mathpar}
\inferrule*[right=evenO] { }{\even~0}
\and  
\inferrule*[right=evenSS]{\even~n}{\even~(S~(S~n))}
\end{mathpar}
Any inhabitant of the type \coqe{even n} must be derivable from these
rules. Inverting an element of an inductive definition is the process
of finding the possible valid derivations of this element. 

Suppose that we have an hypothesis \coqe{even (S x)} for some natural
number \coqe{x}. Since \coqe{S x} cannot be equal to \coqe{0}, we see
that the only rule that may have been apply is the constructor
\texttt{evenSS} hence \coqe{x} is equal to some \coqe{S y} and
\coqe{even y} holds. This corresponds exactly to a case analysis on
the derivations of \coqe{even n} knowing that \coqe{n} must be
matching the pattern \coqe{S x}. 

This is conceptually simple, but the inversion algorithm~\cite{cornes}
implemented in the \coqe{inversion} tactic in such a goal yields a
contrived and big proof term. A concrete manifestation of this problem
for the user is that the proof context becomes cluttered with
seemingly useless equalities, that need to be substituted away by the
user. The mantra \coqe{inversion H; clear H; subst} is both well-known
and despised by long-time Coq users.
 
Monin \cite{monin} described a new technique to build inversion
principle that solves part of the problems of inversion. The idea is
to use dependent pattern matching on the arguments of the inductive
(\coqe{S x} in the above example) to rule out impossible cases, using
a well-crafted ``diagonalization function'' (\coqe{diag} in the
following). Then, this diagonalization function is used as a return
clause in a case analysis on \coqe{H: even (S x)}. For example, if the
conclusion of the proof goal is \coqe{even (pred x)}, we may build the
proof term on Fig.\ref{fig:proof-term}. 

\begin{figure}
  \centering

\begin{multicols}{2}
\begin{coq}
let diag n := 
  match n return Prop with
  | 0 => True
  | S x =>  even (pred x)
  end in
match H as H' in even n' return diag n' 
with 
| eO => I  
| eSS k Hk => _ : even (pred (S k)) end. 
\end{coq}
\end{multicols}
\caption{Proof term for \coqe{even (S x) -> even (pred x)}}
  \label{fig:proof-term}
\end{figure}

We will come back on the precise rules that define the typing of such
dependent elimination in \secref{dependent-pattern-matching}; for the
time being, let's just recall that the return clause %
\coqe{as H' in even n' return diag n'} is a function that binds the arguments of
the inductive (\coqe{S x} as \coqe{n'}), and \coqe{H'} of type
\coqe{even n'}, and which body is the return part. Usually, the return
part is a constant (e.g., \coqe{nat} for the match in the definition
of \coqe{List.length}), but it is not mandatory. In the most general
case, the body part may depend arbitrarily on the type of the
discriminee.
%
Applying Coq's typing rules for the dependent elimination, we get the following types \begin{center}
  \begin{tabular}{l@{\quad}cc}
     & type & type (after reduction) \\
     \hline
    whole match 
    & \lstinline[language=Coq, basicstyle=\normalsize]|diag (S x)| 
    & \lstinline[language=Coq, basicstyle=\normalsize]|even (pred x)| \\
    
    first branch 
    & \lstinline[language=Coq, basicstyle=\normalsize]|diag 0|
    & \lstinline[language=Coq, basicstyle=\normalsize]|True| \\
    
    second branch 
    & \lstinline[language=Coq, basicstyle=\normalsize]|diag (S (S k))|  
    & \lstinline[language=Coq, basicstyle=\normalsize]|even k| \\
\end{tabular}
\end{center}
%
Note that the types in both columns are convertible, we just applied Coq's reduction rules to get the right-hand side column.
%
Looking at these type in order, we see first that the type of the whole match corresponds to the conclusion of the goal, making the whole proof term well-typed.
%
Then, we see that this return clause ``eliminates''the impossible branch \coqe{eO}: the type of this branch is the trivial return type \coqe{True} which is inhabited by \coqe{I}.
%
Finally, we see that the type of the second branch corresponds to the type of the hypothesis \coqe{Hk}, which allows the user to conclude the proof by filling the hole. 


\paragraph{Contributions of the paper.} 
In their inspiring paper, Monin and Shi~\cite{monin-shi} described the process of the creation of those diagonalization functions and state that: 
\begin{quote}
  \em ``[Their] goal is more modest: providing a hand-crafted inversion technique which is flexible enough for the user'' \end{quote} 
%
Indeed, what they provide is a general technique to craft inversions schemes by hand.  In this paper, we mechanize the construction of such diagonalization functions, yielding a new Coq tactic for inversion called \coqe{invert}.
%
In the following, we first introduce some definitions and notations (see \secref{sec:preliminaries}). 
%
Then, we identify a well-defined set of inversions problems that can be solved without introducing any equalities between terms making maximal use of the expressive power of the return clauses (see \secref{}).
%
Next, we proceed to relax the restrictions we impose on inversions problems to tackle the general case of the problem (see \secref{}). 
%
We propose an in-depth comparison of the behavior of our new \coqe{invert} tactic w.r.t. the existing elimination tactics of Coq.   

\end{document}
\verb!http://pages.cs.wisc.edu/~mulhern/Mul2010/paper.pdf!

%%% Local Variables: 
%%% mode: latex
%%% TeX-master: t
%%% End: 
