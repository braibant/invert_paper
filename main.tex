\documentclass{llncs}

\usepackage{macros}
\usepackage{lstcoq}
\usepackage{lstocaml}
\usepackage{mathpartir}
\usepackage{amsfonts}
\usepackage{amsmath}
\usepackage{stmaryrd}
\usepackage{graphics}
\usepackage{array}
\usepackage{tikz}
\usetikzlibrary{shapes, positioning,fit}
\bibliographystyle{plain}
\pagestyle{plain}

\title{Mechanizing small-inversions}
% % \subtitlte{A new elimination tactic to rule them all}
\author{Pierre Boutillier \and Thomas Braibant}
\institute{Inria Paris Rocquencourt}
\begin{document}
\maketitle
\begin{abstract}
  Inductively defined relations are pervasives in the large scale
  formalizations; and many proof steps require to invert an
  hypothesis. Coq's built-in tactic ``inversion'' is made for this
  purpose, but has severe usability and efficiency issues. 
  
  We propose a inversion tactic that mechanizes the ideas behind
  the ``small inversions'' proof-trick \cite{monin} and argue that it
  builds the right elimination principle.  
\end{abstract}
\section*{Introduction}

An inductive definition is often presetend as a set of rules (called
constructors). For example, let us define \coqe{even} by the rules
\newcommand\even{\mathtt{even}}
\begin{mathpar}
\inferrule*[right=evenO] { }{\even~0}
\and  
\inferrule*[right=evenSS]{\even~n}{\even~(S~(S~n))}
\end{mathpar}
Any inhabitant of the type \coqe{even n} must be derivable from these
rules. Inverting an element of an inductive definition is the process
of finding the possible valid derivations of this element. 

Suppose that we have an hypothesis \coqe{even (S x)} for some natural
number \coqe{x}. Since \coqe{S x} cannot be equal to \coqe{0}, we see
that the only rule that may have been apply is the constructor
\texttt{evenSS} hence \coqe{x} is equal to some \coqe{S y} and
\coqe{even y} holds. This corresponds exactly to a case analysis on
the derivations of \coqe{even n} knowing that \coqe{n} must be
matching the pattern \coqe{S x}. 

This is conceptually simple, but the inversion algorithm~\cite{cornes}
implemented in the \coqe{inversion} tactic in such a goal yields a
contrived and big proof term. A concrete manifestation of this problem
for the user is that the proof context becomes cluttered with
seemingly useless equalities, that need to be substituted away by the
user. The mantra \coqe{inversion H; clear H; subst} is both well-known
and despised by long-time Coq hackers.
 
Monin \cite{monin} described a new technique to build inversion
principle that solves part of the problems of inversion. The idea is
to use dependent pattern matching on the arguments of the inductive
(\coqe{S x} in the above example) to rule out impossible cases, using
a well-crafted ``diagonalization function'' (\coqe{diag} in the
following). Then, this diagonalization function is used as a return
clause in a case analysis on \coqe{H: even (S x)}. For example, if the
conclusion of the proof goal is \coqe{even (pred x)}, we may build the
proof term on Fig.\ref{fig:proof-term}. 

\begin{figure}
  \centering

\begin{multicols}{2}
\begin{coq}
let diag n := 
  match n return Prop with
  | 0 => True
  | S x =>  even (pred x)
  end in
match H as H' in even n' return diag n' 
with 
| eO => I  
| eSS k Hk => _ : even (pred (S k)) end. 
\end{coq}
\end{multicols}
\caption{Proof term for \coqe{even (S x) -> even (pred x)}}
  \label{fig:proof-term}
\end{figure}

We will come back on the precise rules that define the typing of such
dependent elimination in \secref{dependent-pattern-matching}; for the
time being, let's just recall that the return clause %
\coqe{as H' in even n' return diag n'} is a function that binds the arguments of
the inductive (\coqe{S x} as \coqe{n'}), and \coqe{H'} of type
\coqe{even n'}, and which body is the return part. Usually, the return
part is a constant (e.g., \coqe{nat} for the match in the definition
of \coqe{List.length}), but it is not mandatory. In the most general
case, the body part may depend arbitrarily on the type of the
discriminee.
%
Applying Coq's typing rules for the dependent elimination, we get the
following types 
%
\begin{center}
  \begin{tabular}{l@{\quad}cc}
     & type & type (after reduction) \\
     \hline
    whole match 
    & \lstinline[language=Coq, basicstyle=\normalsize]|diag (S x)| 
    & \lstinline[language=Coq, basicstyle=\normalsize]|even (pred x)| \\
    
    first branch 
    & \lstinline[language=Coq, basicstyle=\normalsize]|diag 0|
    & \lstinline[language=Coq, basicstyle=\normalsize]|True| \\
    
    second branch 
    & \lstinline[language=Coq, basicstyle=\normalsize]|diag (S (S k))|  
    & \lstinline[language=Coq, basicstyle=\normalsize]|even k| \\
\end{tabular}
\end{center}
%
Note that the types in both columns are convertible, we just applied
Coq's reduction rules to get the right-hand side column.
%
Looking at these type in order, we see first that the type of the
whole match corresponds to the conclusion of the goal, making the
whole proof term well-typed.
%
Then, we see that this return clause ``eliminates''the impossible
branch \coqe{eO}: the type of this branch is the trivial return type
\coqe{True} which is inhabited by \coqe{I}.
%
Finally, we see that the type of the second branch corresponds to the
type of the hypothesis \coqe{Hk}, which allows the user to conclude
the proof by filling the hole.


\subsubsection{Contributions of the paper.} 
In their inspiring paper, Monin and Shi~\cite{monin-shi} described the
process of the creation of those diagonalization functions and state
that:
\begin{quote}
  \em ``[Their] goal is more modest: providing a hand-crafted
  inversion technique which is flexible enough for the
  user'' \end{quote}
%
Indeed, what they provide is a general technique to craft inversions
schemes by hand.  In this paper, we report on the mechanization of the
construction of such diagonalization functions, yielding a new Coq
tactic for inversion called \coqe{invert}.

In the following of the paper, we first introduce some preliminary
definitions and notations (see \secref{sec:preliminaries}); and we
briefly recall how the current Coq inversion tactic works (see
\secref{ssec:inversion}).
%
Then, we identify a well-defined set of inversions problems that can
be solved without introducing any equalities between terms making
maximal use of the expressive power of the return clauses (see
\secref{sec:inversion-pattern}).
%
Then, we proceed to relax the restrictions we impose on inversions
problems to tackle the general case of the problem (see \secref{}).%

Then, we propose a comparison of the behavior of our new \coqe{invert}
tactic w.r.t. the existing elimination tactics of Coq; and evaluate
its usefulness on practical examples (see \secref{}).
%
Finally, we conclude with a comparison with related works and some
perspectives. 

\section{Preliminaries}\label{sec:preliminaries}
In this section, we define the notations and vocabulary we shall use
through the rest of the paper. Here, we settle in the context of the
Coq proof assistant, but we argue that our work could be adapted to
the context of other proof assistants based on type theory, such as
Matita or Agda.

\renewcommand\vector[1]{\overline{#1}} 

First, we let $S$ range over sorts; that is, $S \in \set{\tt Prop,
  Set, Type_{i \in \mathbb{N}}}$ and the meta variable $t$ denote
arbitrary Coq terms. We use the shorthand notation $\forall \vector{x
  : t},e$ to denote the sequence of quantifiers $\forall (x_1:t_1)
... (x_n:t_n),e$ when the precise number of these quantifiers do not
matter.


A Coq inductive definition has the following most general form
\begin{coq}
Inductive name $(z_1: P_1) \dots (z_p: P_p)$ : forall $(a_1:A_1) \dots (a_n:A_n), S$ :=
| $c_1$ : forall $(x^1: \alpha^1_1) \dots (x^{c_1} : \alpha^{c_1}_1)$, name $z_1 \dots z_p$ $t^1_1 \dots t^n_1$
$\vdots$
| $c_m$ : forall $(x^1: \alpha^1_m) \dots (x^{c_m} : \alpha^{c_m}_m)$, name $z_1 \dots z_p$ $t^1_m \dots t^n_m$
\end{coq}
%
First, the inductive definition \coqe{name} has $p$ parameters
$P_1;\dots;P_p$ and $n$ arguments $A_1; \dots;A_n$. We let $arity({\tt name})$
denote the type
%
\coqe{forall $(a_1:A_1) \dots (a_n:A_n), S$}, which is called the
arity of the inductive. We let $C_i$ denote the type of the $i$-th
constructor $c_i$ of \coqe{name}
%
\begin{center}
  $C_i \eqdef$ \coqe{forall $(x^1: \alpha^1_i) \dots (x^{c_i} : \alpha^{c_i}_i)$, name $z_1 \dots z_p$ $t^1_i \dots t^n_i$}
\end{center}
where each $t_i^k$ has type $A_k$.
%
(For the sake of simplicity, we will only consider inductive without
parameters in the following; yet, our algorithms and tactic extend to
the general case.)

Then, we turn to the elimination of these inductives definitions.
Internally, Coq represents a \coqe{match} construct using a more
primitive \coqe{case} construct that takes as arguments the term that
is eliminated $e$, a return clause $R$ and one function $f_i$ per
constructor $c_i$ of the inductive (each $f_i$ corresponds to a
\emph{branch} of the case construct).  The typing rule is expressed as
follows, where the judgement $\Gamma \vdash e : \tau$ reads ``under
typing context $\Gamma$, the expression $e$ has type $\tau$.

\begin{mathpar}
\inferrule*[right=$(\dagger)$]{
  \Gamma \vdash e : \texttt{name}~u_1\dots u_n \and 
%
  \Gamma \vdash R : \forall (a_1:A_1) \dots (a_n:A_n) (H: \texttt{name}~a_1\dots a_n), S'\\ 
  % 
  \Gamma \vdash f_i : \forall (x^1: \alpha^1_i) \dots (x^{c_i} : \alpha^{c_i}_i), R~t^1_i~\dots~t^n_i (c_i~t^1_i~\dots~t^n_i) \quad i \in \set{1,\dots,m}}
%%%%%%%%%%%%%%%%%%%%%%%%%%%%%%%%%%%%%%%%%%%%%%%%%%%%%%%%%%%%%%%%%%%%%%%%%%%%
{\Gamma \vdash {\tt case}~e~{\tt return}~R~{\tt with}~(f_1,\dots,f_n) : R~u_1\dots u_n e} 
\end{mathpar}


(Remark that the sole differences w.r.t. the (user-level) \coqe{match}
syntax are that:
%
(1) the return clause summarizes the \coqe{as ... in ... return ...}
part of the \coqe{match};
%
(2) there is no implicit binders for the arguments of the constructors
in the branches.)

\newcommand\figref[1]{Fig.\ref{#1}}

Let's take an example to clarify these notations. The various elements
of the \coqe{case} construct that corresponds to the \coqe{match} on
the right-hand side of \figref{fig:proof-term} are defined as follows:
\begin{center}
  \begin{tabular}{c@{$\quad\eqdef\quad$}l}
    $e$ & \lstinline[language=Coq, basicstyle=\normalsize]|H| \\
    $R$ & \lstinline[language=Coq, basicstyle=\normalsize]|fun (n' : nat) (H' : even n') => diag n'| \\
    $f_1$ & \lstinline[language=Coq, basicstyle=\normalsize]|I : R 0 eO| \\
    $f_2$ & \lstinline[language=Coq, basicstyle=\normalsize]|fun (k: nat) (Hk: even k) => (_ : R (S (S k)) (eSS k Hk))|
  \end{tabular}
\end{center}
Again, after applying the reduction rules, we have 
\begin{center}
  \begin{tabular}{c@{$\quad\equiv\quad$}l}
    $f_1$ & \lstinline[language=Coq, basicstyle=\normalsize]|I : True| \\
    $f_2$ & \lstinline[language=Coq, basicstyle=\normalsize]|fun (k: nat) (Hk: even k) => (_ : even k)|
  \end{tabular}
\end{center}

\subsubsection{Commutative cuts.}
Let's pause to remark that the typing rule $\dagger$ does not refine
the typing context in the type of the branches. This is the cause of a
frequent problem that we are going to illustrate on an (simplified)
example coming from Chlipala's CPDT~\cite{cpdt}.

\begin{coq}
Lemma lt_0_0 : 0 < 0 -> False := lt_n_0 0. 
Fail Definition pred (n : nat) (H: 0 < n) : nat :=
match n with 
| 0 => match lt_0_0 H in False return nat with end  (* elimination of False *)
| S n => n
end. 
\end{coq}
At this point, Coq complains that the term \coqe{lt_0_0 H} is
ill-typed. Indeed, \coqe{H} has type \coqe{0 < n}. What happens here
is that the typing rule $(\dagger)$ does not modify the typing of the
variables (here \coqe{H}) that occur in the typing context.

A solution to this problem is the pattern known as \emph{commutative
  cuts}. The idea is to refine the type of a variable that occurs in
the context, by generalizing its type in the return clause, and
introducing an extra-abstraction in each branches of the match. In the
example above, the code becomes
\begin{coq}
Definition pred (n : nat) (H: 0 < n) : nat :=
match n as m in nat return 0 < m -> nat with 
| 0 => fun (H : 0 < 0) => match lt_0_0 H in False return nat with end 
| S p => fun (_: 0 < S p) => p
end H. 
\end{coq}
where we interleave the elimination of the inductive and the
introduction of abstractions in the branches. 

Commutative cut are casual when writing dependent eliminations, and
find another use case when one needs to preserve the value of the
arguments of the inductive inside the branches of the match.
\marginpar{Put an example!}

\subsection{Inversion seen as logic programming}\label{ssec:inversion}
The current inversion tactic of Coq was implemented by
Cornes~\cite{cornes}, improving some earlier work by C. Murthy. The
underlying idea is quite simple: given an inductive predicate
\coqe{Ind}, generate an inversion lemma of the form
\begin{displaymath}
  \texttt{inv\_lemma} : \forall \vector{x:t}, \forall (e: {\tt Ind}~\vector{x}), {\tt pred}~\vector{x}
\end{displaymath}
where \texttt{pred} expresses the constraint that $\vector{x}$ must
meet for \coqe{Ind $\vector{x}$} to hold. Typically, this predicate
introduces new variables, and relate the value of these w.r.t. the
arguments $\vector{x}$ using a set of equations. Let's consider the
following example, a definition of the less-or-equal relation.
\begin{coq}
Inductive le : nat -> nat -> Prop :=
  le0 : forall n, le 0 n
| leS : forall n m, le n m -> le (S n) (S m).
\end{coq}
%
Then, to perform an inversion over an hypothesis of the form %
\coqe{le (S n) m}, one can generate (and prove) the following lemma:
\begin{coq}
Lemma inv_lemma n m (H : le n m) : match n with 
                                   | 0 => True
                                   | S n' => exists m', m = S m' /\ le n' m' 
                                   end. 
\end{coq}
Fortunately, the equation \coqe{m = S m'} does not fall in the
category of the ``useless equations generated by \coqe{inversion} and
that need to be substituted by \coqe{subst}''.
%
Yet, in a more general context, the algorithm based on logic
programming that generates the inversion lemmas follows a simple
strategy that is likely to clutter the proof context with redundant
equations.
%
Moreover, the proof term that are generated for these inversion lemmas
are quite big. Since these proof terms are inlined at the call sites
of the \coqe{inversion} tactic, this yields the efficiency issues that
we mentionned in the introduction. 

\section{Inversions without equations}\label{sec:inversion-pattern}
In this section, we identify a subset of inversions problems that can
be solved without introducing equations; and we demonstrate how to
solve them using diagonalization functions.

First, we consider a restricted form of Coq terms with a
\emph{pattern} structure\footnote{that is, terms built recursively using
variables and constructors; not to be confounded with patterns as in ``pattern matching''}
\begin{center}
  \begin{tabular}{ccl@{\quad}l}
    $p$ & ::=& &  {\bf patterns} \\
    & & $x,y$ & variable \\
    & & $c~\vector{p}$ & constructor pattern
  \end{tabular}
\end{center}
%
First, the class of inductive definitions that are used in this
section share the properties that the conclusion of each constructor
of the inductive must be \emph{patterns}: that is, the terms $t$ in
each application ${\tt name}~\vector{t}$ must be patterns.

\newcommand\subst[3]{#1 \left[ #2 \leftarrow #3 \right]}
\newcommand{\mkterm}[5][]{
  \node[term,#1](#2){\begin{tabular}{c}#3\\#4\\#5\end{tabular}};}   
\newcommand{\mkclause}[4][]{
  \node[clause,#1](#2){\begin{tabular}{c@{,}c}#3 & #4 \end{tabular}};}   

\tikzstyle{term} = [draw=red, very thick, rectangle, rounded corners, inner sep=4pt, inner ysep=4pt, node distance=1cm, solid]

\tikzstyle{clause} = [draw=black, very thick, rounded rectangle, inner sep=4pt, inner ysep=4pt, solid]
% rounded rectangle west arc=none,

\tikzstyle{code} = [draw=none]

\begin{figure}[t]
  \begin{multicols}{2}
    \begin{tikzpicture}
      \mkterm{empty-left}{$ \tt []$}{$\tt ty$}{$\tt concl $}
    % 
      \node [right=0.0cm of empty-left] (arrow){$\Rightarrow$};
    % 
      \node[term, dashed, right=0.0cm of arrow] (empty-right){%
        {$\tt concl $} };
      % inferrule style
      \node[draw=none, fit=(empty-left) (empty-right)](bb-bottom){};
      
      \draw [thick] (bb-bottom.north west) -- (bb-bottom.north east) node[right] {\sc Emp};
    \end{tikzpicture}
    \begin{tikzpicture}
      \mkterm{var-left}{$ \tt y :: q$}{$\tt \forall (x:t_x),
        ty$}{$\tt concl$}
    % 
      \node [right=0.0cm of var-left] (arrow){$\Rightarrow$};
    % 
      \node[term, dashed, right=0.0cm of arrow] (var-right){%
        \begin{tikzpicture}
          \node[code](bout){$\tt fun~(x:t_x) \Rightarrow $};
          \mkterm[right=0.0cm of bout]{}{$ \tt q$}{$\tt
            \subst{ty}{x}{y}$}%
          {$\tt \subst{concl}{x}{y}$}
        \end{tikzpicture}
      };
    
      % inferrule style
      \node[draw=none, fit=(var-left) (var-right)](bb-bottom){}; \draw
      [thick] (bb-bottom.north west) -- (bb-bottom.north east)
      node[right] {\sc Var};
    \end{tikzpicture}
  \end{multicols}
  % \newcommand{\highlight}[1]{%
  %   \tikz[baseline,anchor=base]{
  %     \node[text centered, anchor=text,
  %     text height=1.5ex, 
  %     inner sep=2pt, rectangle, draw=black, sharp corners, solid, fill=black!10 ] {#1};
  %   }
  % }

  \newcommand{\highlight}[1]{#1}
  \begin{center}
    \begin{tikzpicture}
      \mkterm{constr-left}%
      {$\tt c~\vector{p}::q$}%
      {$\tt \forall (x:t_x), ty$}%
      {$\tt concl$}
    % 
      \node [right=0.0cm of constr-left] (arrow){$\Rightarrow$};
    % 
      \node[term,dashed, right=0.0cm of arrow] (constr-right){%
        \begin{tikzpicture}
          \node[code](lambda){$\tt fun~(x:t_x) \Rightarrow $};
          \node[code, right=0.0cm of lambda](header){$\tt
            case~x~return~\highlight{clause}~with$};
        
          \node[matrix, below=0.0cm of header.west, anchor=north
          west](branches) { \node[code]{$| {\tt c} \Rightarrow$}; &
            \mkterm{body}{$\tt \vector{p}~@~args~@~q$}{$\forall (\vector{x}:\vector{\alpha}), {\tt clause}$}{$\tt concl$} \\
            \node[code]{$| \_ \Rightarrow $}; &
            \node[code]{$\tt I$}; \\
          }; \node[draw=none, below=-0.4cm of branches.south west,
          anchor=north west]{ $\tt end~\highlight{args}$};
        \end{tikzpicture}
      }; \node[draw=none, fit=(constr-left)
      (constr-right)](bb-bottom){}; \draw [thick] (bb-bottom.north
      west) -- (bb-bottom.north east) node[right] {\sc Constr};
      % the premises of the inference rule
      \node[matrix, above=0.0cm of bb-bottom.north]  { 
        \mkclause[anchor=center]{}{$\tt t_x$}{$\tt ty$} & 
        \node [anchor=center] {$\Rightarrow$}; & 
        \node[clause, anchor=center, dashed] {$\tt clause, args$}; \\
      };
    \end{tikzpicture}
  \end{center}
  \caption{Diagonalization algorithm (building cases)}\label{fig:mkterm}   
\end{figure}

\begin{figure}[t]
  \centering  
  \begin{tikzpicture}
    \node[matrix, align=center](concl){ 
      \mkclause [anchor=center]{}{$\tt Ind~\vector{a}$}{$\tt ty$} &
      \node [anchor=center] (arrow) {$\Rightarrow$};  &
      \mkclause[anchor=center,dashed]{}{$\tt clause$}{$\tt args$} \\      
    };
    %%%%%%%%%%%%%%%%%%%%%%%%%%%%%%%%%%%%%%%%%%%%%%
    \node(premises)[matrix, above=0.0cm of concl.north, align=center ]{
      \node [anchor=center](prepare) {$\tt args, ty' \eqdef$ \sc make-commutative-cuts($\tt \vector{a}, ty$)}; &       
      \node [draw=none, minimum width=1cm]{};
      &
      \mkterm[anchor=center]{foobar}{$\tt \vector{a}$}{$arity({\tt Ind})$}{ty'} & 
      \node [anchor=center] (arrow) {$\Rightarrow$};  &      
      \node [term,anchor=center, dashed](clause) {$\tt clause$}; &  \\
    };
    \draw [thick] (premises.south west) -- (premises.south east) node[right] {\sc Ind};    
  \end{tikzpicture}
  \caption{Diagonalization algorithm (building return clauses)}\label{fig:mkclause}
\end{figure}

The construction of the diagonalizer for a given inversion problem
interleaves three basic procedures. In the following, we describe
these procedures using block diagrams and inference rules; and we take
the convention that plain boxes correspond to procedure calls and
dashed boxes encapsulate the corresponding results.

The first procedure is called {\tt mkterm}; it is described in
\figref{fig:mkterm} and is used to build a cascade of \coqe{case}
constructs. 
%
It takes as arguments a list of patterns $l$, a type $\tt ty$, and a Coq term $\tt concl$.
%
In an initial call to {\tt mkterm}, this list of pattern corresponds
to the arguments of inductive that is inverted; the type corresponds
to the arity of the inductive that is inverted; and the term $\tt
concl$ correspond to the conclusion of the goal. (Intuitively, the
list of pattern and the types could be zipped together, but it would
make for less idiomatic data structures when it comes to the real
implementation.)

The first interesting case is when the list of patterns starts with a
variable (rule {\sc Var}): in that case, we generalize this variable
in both the conclusion and the return clause.

The second interesting case is when the list of patterns starts with a
constructor {\tt c} applied to a list of patterns ${\tt \vector{p}}$.
%
The variable {\tt x} corresponds to the generalization of this
argument and we use a suitable elimination

\section{Related works}
\begin{itemize}
\item Investigate to what extent this ``pattern linear'' class matches
  something in Conor's paper.
\item Monin
\item Cornes
\item Reationship with the typing of GADTs in OCaml
\item Adam state that inferring the return clauses in match is not
  decidable. Why are we that efficient here ?
\end{itemize}

\end{document}
\verb!http://pages.cs.wisc.edu/~mulhern/Mul2010/paper.pdf!

%%% Local Variables: 
%%% mode: latex
%%% TeX-master: t
%%% End: 
